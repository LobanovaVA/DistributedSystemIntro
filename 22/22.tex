\newpage
\section {Билет 22. Взаимное исключение. Решение на основе сервера. Алгоритм Дейкстры с общими регистрами. Алгоритм Петерсона для двух процессов. Алгоритм на основе голосования (Maekawa).}

\begin{center}
    \hypertarget{b_22_link}{\textit{\underline{Задача взаимного исключения.}}}
\end{center}

\textit{Постановка задачи:} \\
Задано N процессов, конкурирующих за общий ресурс. Каждый процесс циклически выполняет код, принадлежащий и не принадлежащий \textbf{критической секции (CS)}(т.е. иногда процесс заходит в критическую секцию и потом из неё выходит).
Задача стоит в разработке алгоритма, который гарантирует, что в каждый момент времени в критической секции находится не более одного процесса.

\textit{Требования:} 
\begin{itemize}
\item Основные:
\begin{itemize}
\item \textbf{Safety}: не более одного процесса находится в критической секции
\item \textbf{Liveness}: если один или несколько процессов хотят зайти в CS, то рано или поздно по крайней мере один из них зайдет в критическую секцию (при условии, что время нахождения процессов в CS ограничено)
\end{itemize}
\item Дополнительные:
\begin{itemize}
\item \textbf{Отсутствие голодания}: если процесс $p$ хочет войти в критическую секцию, то рано или поздно он войдет в критическую секцию \footnote{\textit{прим. авт.:} из Liveness это условие не следует} 
\item \textbf{Справедливость}: если процесс $p$ захотел зайти в CS раньше процесса $q$, то он зайдет в неё раньше
\end{itemize}
\end{itemize}

Задача ставится не только для систем с распределенной памятью, но и для систем с общей памятью: процессы могут взаимодействовать через общие переменные, атомарны только операции чтения или записи. Далее приложены описания нескольких алгоритмов для решения задачи на 2ух или более процессах. 


\begin{algorithm}
\caption{Решение на основе сервера}
\label{algServer}
Решение задачи взаимного исключения для $N$ процессов

\textit{Суть работы}: сначала вызывается алгоритм выбора лидера и выбирается один процесс лидер (пусть его номер $seq$). Далее для всех процессов кроме лидера будет единый алгоритм.

\textit{Требования}: не удовлетворяет требованию справедливости
\begin{algorithmic}
\small
\State У процессов, которые не являются лидерами есть две функции
\Procedure{try:}{} \Comment{попытка входа в CS}
  \State send (seq, <enter, i>) \Comment{seq - идентификатор лидера, i - номер запрашивающего процесса} 
  \State wait (<allow>) \Comment{i-ый процесс ждёт разрешения}
  \State IN CS \Comment{попал в критическую секцию}
\EndProcedure \\

\Procedure{exit:}{} 
  \State send (seq, <exit, i>) 
\EndProcedure
\\
\State Лидер должен уметь обрабатывать запросы, которые к нему приходят:
\Procedure{init:}{}  \Comment{запуск сервера}
  \State queue = []
  \State current = None \Comment{CS свободна}
\EndProcedure \\

\Procedure{on}{<enter, i>} \Comment{обработка запроса на вход в CS}
  \If{current == None}
    \State send (i, <allow>) 
  \Else
    \State queue += i
  \EndIf  
\EndProcedure \\

\Procedure{on}{<exit, i>}  \Comment{обработка запроса на выход из CS}
  \If{queue is not empthy}
    \State q = queue.getFirst ()
    \State current = q
    \State send (q, <allow>)
  \Else
     \State current = None
  \EndIf
\EndProcedure


\end{algorithmic}
\end{algorithm}

\begin{algorithm}
\caption{Алгоритм Деккера}
Решение задачи взаимного исключения для двух процессов

\textit{Требования:} не удовлетворяет требованию справедливости
\label{algDekker}
\begin{algorithmic}
\Ensure  \Comment{общие данные}
\\$status[2]$ \Comment{массив из двух чисел. Для чтения каждому из процессов. Для записи у $i$-ого процесса есть доступ только к $i$ ячейке.}\\ 
$turn$ \Comment{переменная, в которую оба процесса могут записывать и читать данные}\\ 
\While{$status[other] == competing$} 
  \If{$turn == other$} 
    \State $status[i] = out$
    \State $wait until (turn == i)$
    \State $status[i] = competing$
  \EndIf
\EndWhile

\State \textbf{CS}
\State $turn = other$
\State $status[i] = out$
\end{algorithmic}
\end{algorithm}


\begin{algorithm}
\caption{Алгоритм Дейкстры}
Решение задачи взаимного исключения для $N$ процессов

\textit{Требования:} не удовлетворяет требованию справедливости

\label{algDijkstra}
\begin{algorithmic}
\Ensure \\$status[i] \in \{competing, out, cs\}$ \\
$turn \in \{1,...N\}$ 

\Repeat
  \While{$turn \neq i$} 

    \If{$status[turn] == out$} 
      \State $turn := i$
    \EndIf
  \EndWhile
  \State $status[i] = cs$
\Until{$\nexists other:\ status[other] = cs$}
\State \textbf{CS}
\State $status[i] = out$
\end{algorithmic}
\end{algorithm}

\begin{algorithm}
\caption{Алгоритм Петерсона}
Решение задачи взаимного исключения для 2 процессов: 

\textit{Суть работы:} Перед тем как начать исполнение критической секции кода, процесс должен вызвать процедуру LOCK() со своим номером в качестве параметра. Она должна организовать ожидание процессом своей очереди входа в критическую секцию. После исполнения критической секции и выхода из неё процесс вызывает другую процедуру UNLOCK(), после чего уже другой процесс сможет войти в критическую область.
\label{algPeterson}
\begin{algorithmic} 
\Ensure \\$want[2]$ \Comment{массив из двух чисел}\\ 
$victim$ \Comment{переменная, в которую оба процесса могут записывать и читать данные}\\ 
\Procedure{lock:}{}
  \State $want[i] = true$  
  \State $victim = i$
  \While{$want[1-i] == true$ \textbf{and} $victim == i$}
    \State pass
  \EndWhile
\EndProcedure

\\
\Procedure{unlock:}{}
  \State $want[i] = false$
\EndProcedure
\end{algorithmic}
\end{algorithm}